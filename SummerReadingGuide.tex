\documentclass[12pt]{article}

\author{Peter Strawn}

\begin{document}
\begin{center}
\Large{\textbf{\textit{D is for Digital}}}

\large{\textbf{Summer Reading Guide}}
\end{center}

This book is a fascinating exploration of all the areas that we will cover in greater detail during the school year. I don't expect you to remember everything or understand everything the first time through. That said, you should find this material interesting and engaging. If you think it's amazing that \textit{everything} we see and do on a computer comes down to sequences of 0s and 1s processed millions of times per second, then you will enjoy this read and the class.

You will often hear me use the phrase ``under the hood'' next year; this book is the perfect way to begin looking under the hood at a computer and understanding how the technology that we so often take for granted actually works. I hope you truly enjoy this read. As you progress through the chapters, keep in mind the questions below. Mark their answers. Jot notes in the margins. Think of this as an annotation guide. You will not have to turn anything in, but do plan on some sort of assessment the first week of class.

Enjoy this initial dive into the world of computers and computing. I look forward to our journey together in CS50 AP in the fall.\footnote{I am typesetting this document using a language called \LaTeX{} (pronounced ``lay-tech'' or ``lah-tech''). You will see documents like this throughout our year together. Its overall aesthetics are designed to make documents both more visually appealing and, more importantly, more readable vis-\`{a}-vis eye strain. As an example, consider all the room you have for annotations in each margin and the spacing around and within each section of questions.} 

\section*{Preface}
\begin{itemize}
	\item What are the three core technical areas of computing? (Hint:\  they form the three parts of this book.)
	\item What is a potential fourth, according to Kernighan?
\end{itemize}

\section*{Introduction}
\begin{itemize}
	\item What is one of the many significant technical innovations of the 2000s that Kernighan explains? How did it change and/or improve upon previous technology?
	\item Kernighan identifies three important underlying ideas to how computers and communications systems work. What are they?
	
\end{itemize}

\hrulefill

\section*{Part I:\ Hardware}
\begin{itemize}
	\item Who is often referred to as the world's first programmer? What is she known for?
\end{itemize}

\subsection*{What's in a Computer?}
\begin{itemize}
	\item What is the role of the CPU?
	\item If you have a 1.7 GHz computer, what does the ``1.7 GHz'' part mean?
	\item What does ``mega--'' mean? What does ``giga--'' mean?
	\item What is the role of RAM? What information does it store?
	\item What is the difference (or trade-off) between storing something on a disk versus storing something in RAM?
	\item The most important element of a computer's electronic circuitry is a logic gate. What is a logic gate? (Check out the video for logic gates in the Crash Course Computer Science series on YouTube.)\footnote{DFTBA!}
	\item What is Moore's Law?
\end{itemize}

\subsection*{Bits, Bytes, and Representation of Information}
\begin{itemize}
	\item What are the three fundamental ideas about how computers represent information?
	\item What is the difference between ``analog'' and ``digital''?
	\item What is a pixel?
	\item The iPhone 7 has a 12-megapixel camera and a screen with 1334-by-750-pixel resolution at 326 ppi (pixels per inch). Based on your understanding of a pixel, what do these tech specs mean in simple terms?
	\item Chances are that you have at least one digital file of music on your computer. See if you can find out its bit rate and sample rate. What are they? (In iTunes, go to ``Edit $>$ Get Info $>$ File.'')\footnote{This information is called metadata, a key concept we will address in the course, especially when it comes to music and pictures.}
	\item What is ASCII? What is Unicode?
	\item Unicode makes emojis possible. See if you can find the most recently created emojis. What are their Unicode values? (Bonus: how many bits/bytes does it take to represent one Unicode character? How does this contrast to ASCII?)
	\item What is a ``bit''?
	\item Why would a power switch have a 0 and a 1 on it? What state does each represent?
	\item If you have \textit{N} bits, how many different patterns of 0s and 1s can you represent?\footnote{At this point the material might start to seem a little more complex. That's okay for now. Absorb what you can. We will spend a lot of time going over the binary system in class.}
	\item Let's say you have a 1-terabyte hard drive. This could mean two different things (controversially). What are the two possible values for 1-terabyte based on powers of two and powers of ten, respectively?
	\item What is a ``byte''? How many different values can be encoded in 1 byte?
	\item What is the hexadecimal number system?\footnote{This information can be even more complex. Again, understand what you can. We will spend time on it in class. Just be familiar with the concept of hexadecimal numbers.}
	\item Where is the most common place to come across hexadecimal?
	\item Kernighan notes a ``critical thing'' near the end of section 2.3. Mark this paragraph. It is an idea fundamental to this course. What is his point here?
\end{itemize}

\subsection*{Inside the CPU}
\begin{itemize}
	\item What is the purpose of an accumulator?
	\item Kernighan's explanation of the ``Toy'' computer will be a helpful example when it comes time for us to write and understand code in the C programming language. There are also helpful computational thinking ideas in this section. For example, what is the purpose of ``\texttt{GOTO}''? ``\texttt{IFZERO}''?
	\item The bottom of page 39 has some essential ideas for how to construct programs. What key ideas emerge here?\footnote{Notice how you don't need to program to get better at the logic of programming; the key to being a good programmer is not knowledge of syntax but logical and ordered thinking.}
	\item In very simple terms, what is the fetch-decode-execute cycle?
	\item In the context of computer memory, what purpose do caches serve?
	\item Kernighan gives you a fun Google trick to try at the end of 3.3. Come up with an obscure, seemingly random phrase, and note the time it takes Google to respond. You'll see this at the top of the results. Then, run it again immediately. What is the time differential? Caching at work!\footnote{I Googled the phrase ``hippopotamus homeric coffee.'' The first search took 1.03 seconds, the second only 0.41. See how high you can get the response time.}
	\item Who was Alan Turing?\footnote{You may also know him as the character played by the inimitable Benedict Cumberbatch in \textit{The Imitation Game}.} What is the ``Turing test''?
	\item What does CAPTCHA stand for?
\end{itemize}

\hrulefill

\section*{Part II:\ Software}
\begin{itemize}
	\item What is software? How is it different from hardware?
\end{itemize}

\subsection*{Algorithms}
\begin{itemize}
	\item How does Kernighan define ``algorithm''?
	\item Kernighan gives a brief aside to data structures. What is his short explanation for this term?
	\item What does the term ``linear-time'' have to do with algorithms?
	\item What is binary search? What real world application does it have?
	\item Sorting algorithms will be a focus of our study later in the first semester. For now, which algorithm is better? In simple terms, how does this better algorithm work? That is, what is its general approach to sorting?
	\item What is the Traveling Salesman Problem?
	\item Look up the word ``heuristic'' in the context of computer science. How does it relate to the end of 4.4?
\end{itemize}

\subsection*{Programming and Programming Languages}
\begin{itemize}
	\item What is the difference between an algorithm and a program?
	\item What is the function of an assembler and assembly language?
	\item Are assembly language instructions the same for all computer processors?
	\item What is the function of a compiler?
	\item What are the advantages of high-level languages relative to assembly language?
	\item What are the five high-level languages Kernighan demonstrates code with? Which seems hardest to read and understand? Easiest?\footnote{For your reference, CS50 uses C and Python as its primary languages. Other than a quick introduction to programming using Scratch, all of first semester will be spent focused on learning C.}
	\item What is a library? What is a library's API?
	\item What is the term for a flaw in one's code? What is the origin of this term as it relates to computers?
	\item What are general definitions for the following terms: trade secrets, copyright, patents, licenses?
	\item What is a ``standard'' in the world of computing?
	\item What is the relationship between source code and object code?
	\item What does it mean for code to be ``open source''?
	\item What is one example of open source software? (Hint: I'm typing this on a Chromebook, which relies significantly on open source.)
\end{itemize}

That's all I want you to read: 5 chapters. Really. Stop at page 83 (unless you simply can't put it down in which case read as much as you like). That's about the first $1/3$ of the book. We will use the latter 7 chapters throughout the school year. Because of its relatively short length as summer reading assignments go, don't read this just once. It is certainly dense and introduces many concepts which may be brand new to you. I encourage lots of frequent Googling. Dive down some Wikipedia rabbit holes. Explore the tech specs of your own computer(s). (For example, what is the size of your L1 cache?) Read about the history of Linux (which our CS50 IDE\footnote{IDE stands for Integrated Development Environment. In simple terms it is where we write and execute our programs.} uses as its operating system). These 5 chapters will prepare us to jump right into Unit 0 on Day 0 in August.\footnote{Computer scientists start counting at 0, not 1.}

As I said above, I cannot wait to start CS50 AP with all of you. Enjoy your summer. Learn a little in between long bouts of sleep and Netflix. See you in the fall!

\end{document}